\documentclass[article]{memoir}

\usepackage{amsmath}

\title{Notes from TEK5050}
\author{Kristian Monsen Haug}
\date{\today}

\begin{document}
\maketitle


\chapter{Introduction}

Not very important?

\chapter{Light and radiometry}
\section{Crash course: Light}

Three models of light:
\begin{enumerate}
    \item Ray model: Useful for modeling light geometrically.
    \item Wave model: Works where the ray model breaks down (diffraction and interference)
    \item Particle model: Describes how light is generated and absorbed. Light is described as \emph{photons}. Useful when talking about noise properties of detectors.
\end{enumerate}

\paragraph{Maxwell's equations}
\begin{equation}
    \frac{c^2}{\varepsilon \mu} \nabla^2 \mathbf{E} = \frac{\partial^2\mathbf{E}}{\partial t^2}    
\end{equation}

\section{Crash course: Interaction of light with matter}

\paragraph{Energy flux}
The rate of energy transfer. Both units \([W]\) and \([Wm^2]\) can be used, depending on context(?)

\paragraph{Refraction}
occurs when light travels from one (transparent?) medium to another. The change in propagation speed leads to a bending of the light described by Snell's law:
\begin{equation}
    n_1 \sin \theta_1 = n_2 \sin \theta_2
\end{equation}
Where \(n\) is the refraction index, and \(\theta\) is the angle

\paragraph{Absorption}
happens when light loses energy while propagating through a medium. Energy is transferred to the medium. Often as heat, but also occurs in generation of power from a solar cell. The degree of absorption depends on the wavelength of the light, and the material. Over a distance \(x\), the resulting flux is given by
\[
    \phi = \phi_0 \exp{(- \alpha x)}
\]
where \(\alpha\) is the absorption coefficient. We also denote \(1/\alpha\) as the \emph{absorption length}

\paragraph{Reflection and transmission}
When incident on a smooth surface, light is reflected. Often, some is also transmitted through the angle. The relationship (angle and energy) between the incident, reflected and transmitted waves  are given by Fresnel's formulas.

% TODO: Write Fresnel's formulas
The left hand side of the formulas give the amplitude reflection and transmission coefficients (relative amplitude of the three waves), depending on parallel or perpendicular to the plane of incidence.


\end{document}


