\documentclass[article]{memoir}

\usepackage{amsmath}

\newcommand{\unit}[1]{\left[#1\right]}

\title{Notes from TEK5050}
\author{Kristian Monsen Haug}
\date{\today}

\begin{document}
\maketitle


\chapter{Light and interaction with matter}

Three models of light:
\begin{enumerate}
    \item Ray model: Useful for modeling light geometrically.
    \item Wave model: Works where the ray model breaks down (diffraction and interference)
    \item Particle model: Describes how light is generated and absorbed. Light is described as \emph{photons}. Useful when talking about noise properties of detectors.
\end{enumerate}

\paragraph{Maxwell's equations}
\begin{equation}
    \frac{c^2}{\varepsilon \mu} \nabla^2 \mathbf{E} = \frac{\partial^2\mathbf{E}}{\partial t^2}    
\end{equation}

\paragraph{Energy flux}
The rate of energy transfer. Both units \([W]\) and \([Wm^2]\) can be used, depending on context(?)

\paragraph{Refraction}
occurs when light travels from one (transparent?) medium to another. The change in propagation speed leads to a bending of the light described by Snell's law:
\begin{equation}
    n_1 \sin \theta_1 = n_2 \sin \theta_2
\end{equation}
Where \(n\) is the refractive index, and \(\theta\) is the angle

When light is refracted, the speed and wavelength changes, but the frequency stays the same.

\paragraph{Refractive index}
\[
    n = \sqrt{\varepsilon \mu} = \frac{\text{speed of light in vacuum } c}{\text{speed of light inside the material}}
\]
\paragraph{Absorption}
happens when light loses energy while propagating through a medium. Energy is transferred to the medium. Often as heat, but also occurs in generation of power from a solar cell. The degree of absorption depends on the wavelength of the light, and the material. Over a distance \(x\), the resulting flux is given by
\[
    \phi = \phi_0 \exp{(- \alpha x)}
\]
where \(\alpha\) is the absorption coefficient. We also denote \(1/\alpha\) as the \emph{absorption length}

\paragraph{Reflection and transmission}
When incident on a smooth surface, light is reflected. Often, some is also transmitted through the angle. The relationship (angle and energy) between the incident, reflected and transmitted waves  are given by Fresnel's formulas.

% TODO: Write Fresnel's formulas
The left hand side of the formulas give the amplitude reflection and transmission coefficients (relative amplitude of the three waves), depending on parallel or perpendicular to the plane of incidence.

\paragraph{Diffuse refraction and volume scattering}
When light is incident on a rough surface, light is reflected in many different directions. The way this happens, is described by the material's bidirectional reflectance distribution function (BRDF). This must be found experimentally.

This sort of ``random'' scattering can also happen \emph{inside} materials, where small parts of the beam is reflected in different directions, due to inhomogeneities in the material.

\chapter{Radiometry}
Subfield of optics concerned with measuring the amount of light.

\paragraph{Solid angle} Just like we measure angle in terms of the distance it spans when projected onto the unit circle, we measure solid angle in terms of the area on the unit sphere.

\paragraph{Radiance}
\[
L = \frac{\text{light energy}}{\text{surface area} \times \text{time} \times \text{solid angle}} \left[\frac{J}{m^2 s\; \text{steradian}}\right]
\]
As a practicau example, consider a sheet of paper of a certain size. In the equation above, we use the area of the sheet. The radiance also depend on the solid angle it spans as seen from our eye.

\paragraph{Irradiance and excitance}
Total amount of power per surface area
\[
    E\text{ or } M = \iint_\Omega L d\Omega \left[\frac{J}{m^2 s}\right]
\]
\paragraph{Flux}
Integrating irradiance over e.g. the area of a detector, gives us the total power over an area.
\[
    \phi = \iint_\text{area} E dA \left[W\right]
\]
\paragraph{Energy}
\[
    Q = \int_\text{time} \phi dt [J]
\]

\paragraph{Intensity}
It is sometimes useful to consider light sources as point (e.g. stars). To obtain the same measure of light strength for other sources, we need to integrate over the source area 
\[
    I = \iint_{\text{source area}} L dA \unit{\frac{J}{s \text{steradian}}}
\]

\paragraph{Spectral radiance}
``Spectral'' variants of the above densities are defined because we sometimes are interested in differentiating between different wavelength.
\[
    L_{\lambda} = \frac{\text{light energy}}{\text{surface area} \times wavelength \times \text{time} \times \text{solid angle}} \left[\frac{J}{m^2 \;\mu m \; s\; \text{steradian}}\right]
\]

The unit \emph{flick} is sometimes used (1 \(W\) per steradian per square centimeter of surface per micrometer of wavelength). The radiance \(L\) can be obtained by simply integrating over every wavelength.

\paragraph{Quantum measures}
Instead of measuring the energy, we are often interested in counting the number of photons. We precede their names with ``photon'' and use the subscript \(q\). Thus, we obtain e.g. ``photon spectral radiance'' \(L_{q\lambda}\)

\paragraph{Lambertian source}
A Lambertian source is an ideal diffuse reflector: The radiance of the reflected light is independent of angle (brightness is the same regardless of viewing angle). Consequently, the total amount of light varies with angle proportionally to the projected area in the viewing direction.

The intensity is
\[
    I = \frac{\partial \phi}{\partial \Omega} = L A_{\text{source}} \cos{\theta}
\]
Using this, we can obtain a simple relation between radiance and excitance
\[
    M = \iint_{\text{hemisphere}} L \cos{\theta} d \Omega = L \pi
\]

\paragraph{Planck's law}
Expressed as different quantities
\begin{itemize}
    \item Spectral radiance
	\[
	    L_\lambda = \frac{2hc^2}{\lambda^5} \frac{1}{e^{hc/\lambda k T} - 1} \unit{\frac{W}{m^2 m \; sr}}
	\]
	
    \item Photon spectral radiance

	\[
	    L_{q\lambda} = \frac{2c}{\lambda^4} \frac{1}{e^{hc/\lambda k T} - 1} \unit{\frac{\text{photons}}{s \;m^2 m \; sr}}
	\]
    \item Spectral excitance
	\[
	    M_{\lambda} = \frac{2\pi h c^2}{\lambda^5} \frac{1}{e^{hc/\lambda k T} - 1} \unit{\frac{W}{m^2 m}}
	\]
	

    \item Photon spectral excitance
	\[
	    M_{q\lambda} = \frac{2\pi c}{\lambda^4} \frac{1}{e^{hc/\lambda k T} - 1} \unit{\frac{\text{photons}}{s m^2 m}}
	\]
\end{itemize}

\paragraph{The Stefan--Boltzmann law}
Total power per area for a black body
\[
    M(T) = \frac{2 \pi}{15} \frac{k^4}{c^2 h^3}T^4 \equiv \sigma T^4 \unit{\frac{W}{m^2}}
\]
\paragraph{The Wien displacement law}
Tells us the peak wavelength for a black body of a certain temperature, according to Planck's law.
\[
    \lambda_{\text{max}} = \frac{2898}{T} \approx \frac{3000}{T} \unit{\mu m}
\]
\paragraph{Emissivity}
Tells us how much something radiates, compared to an ideal black body
\[
    \varepsilon(\lambda, T) = \frac{M_\lambda(\lambda, T)_{\text{source}}}{M_{e\lambda}(\lambda, T)_{\text{Planck}}}
\]

We can use this with the Stefan--Boltzmann law
\[
    M(T) = \varepsilon \sigma T^4
\]
By conservation of energy
\[
    \phi_{\text{absorbed}} +\phi_{\text{reflected}} +\phi_{\text{transmitted}} =\phi_{\text{incident}}
\]
We obtain Kirchoff's law
\[
    A + R + T = 1
\]
which in turn gives us a relation between the emissivity and the reflectance
\[
    \varepsilon = A = 1 - R
\]

For opaque objects.

% TODO: Some practical light sources and atmosphere

\chapter{Detectors}

\paragraph{Photon detectors}
\begin{itemize}
    \item ``Counts'' photons. Incoming photons excite an electron.
    \item Photons must have a certain energy (cutoff frequency)
    \item Respond proportionally to photon flux. Higher frequency does not mean higher output.

\end{itemize}
\paragraph{Thermal detectors}
\begin{itemize}
    \item No cutoff frequency.
    \item Photons hitting the detectors allow the excited state to relax, which generates heat we can detect.
\end{itemize}

\section{Signal model}
Assume input irradiance \(E\) with wavelength \(\lambda\) onto a photon detector with area \(A_d\). Output \emph{photocurrent} \(I_{ph}\) consisting of excited electrons.

\paragraph{Photon flux}
\[
    \phi_q = \frac{\text{flux}}{\text{photon energy}} = \frac{EA_d}{hv/\lambda} = \frac{EA_d\lambda}{hc}
\]
\paragraph{Photocurrent}
In the ideal case, all photons that hit the detector, excite one electron which in turn contributes to the current, given by
\[
    I_{ph} = q \phi_q = q E_q A_q = \frac{q E A_d \lambda}{hc}
\]

\paragraph{Quantum efficiency}
There is not necessarily a one-to-one correspondence between incoming photons, and output photoelectrons, as not all incoming photons contribute. They might be reflected, pass through the detector, or be absorbed somewhere that doesn't contribute to the stream of electrons. Also, the excited electrons can relax before contributing to the photocurrent. We measure the detector's efficiency in terms of
\[
    \eta = \frac{\text{output photoelectron rate}}{input photon arrival rate}
\]
A more realistic model of the photocurrent therefore is
\[
    I_{ph} = q \eta(\lambda) \phi_q
\]
\paragraph{Responsivity}
\[
    R(\lambda) = \frac{\text{output signal}}{\text{incoming flux}} = \frac{I_{ph}}{EA_d} = q \eta(\lambda) \frac{\lambda}{hc}
\]

The responsivity to a spectrum \(\phi_e(\lambda)\)
\[
    R = \frac{\int_0^\infty R(\lambda)\phi_{e}(\lambda)\; d\lambda}{\int_0^\infty \phi_e(\lambda)\; d\lambda}
\]
If the spectrum is a Planck spectrum with temperature \(T\), \(R(T)\) is called the \emph{blackbody responsivity} at temperature \(T\).

This assumes that the detector responds linearly to the amount of light. For modern detectors, this is often the case. However, it might not always be true. Film and eyes are not linear.

\paragraph{Frequency dependence of responsivity (time constant)}
The output signal often cannot represent the incoming photons accurately, because the frequency the photons arrive with (photon flux) is higher than what can propagate as an electric signal. We can approximate this with a time constant
\[
    I_{ph}(t) = \frac{\eta(\lambda)q\lambda}{hc}\phi\left(1 - e^{-t/\tau}\right) \quad t > 0
\]
where the flux \(\phi\) turn on at \(t=0\).

For AC, we get
\[
    |i_{\text{sig}}| = \frac{\lambda q}{h c} \frac{\eta(\lambda) \phi_{sig}}{\sqrt{1 + (2\pi f \tau)^2}}
\]
And the responsivity is
\[
    R_i(\lambda, f) = \frac{i_{\text{sig}}}{\phi_{\text{sig}}} =  \frac{\lambda q}{h c} \frac{\eta(\lambda) }{\sqrt{1 + (2\pi f \tau)^2}}
\]
\paragraph{Dark Current}
Current from the detector that is not caused by incoming light. Heavily dependent on temperature and voltage across the detector. Impossible to get rid of, but can be measured by blocking out the incoming light, and compensated for in the processing.

\paragraph{Integration and sampling}
Instead of sampling the output analog signal from the detector at regular intervals, we often perform piecewise integration of the signal, and store the integrated values. In other words, we ``count'' the photons received over a piece of time.
\[
    n_e = \frac{Q}{q} = \frac{I_{ph} t_i}{q} = \frac{\eta \phi \lambda t_i}{hc} = \eta \phi_q t_i
\]
where \(Q\) is the integral of the photocurrent over a time step \(t_0\) to \(t_i\).
\[
    Q = \int_{t_0}^{t_0+t_i} I_{ph} \; dt
\]
The time step is known as the \emph{integration time}, or \emph{exposure time} in photography.

In practice, the integration is done with capacitors, which will have a voltage proportional to the integral. At the end of the time step, the voltage is read, and the capacitor reset. 

\section{Signal theory}

\paragraph{Square law}
The energy carried by a signal is proportional to the square of the amplitude.
We write
\[
    \phi = \sqrt 2 \phi_{\text{sig}}(1 + \sin({\omega t}))
\]
where \(\phi_{\text{sig}}\) is the RMS/effective amplitude.
Mean flux, or optical power, is given by
\[
    \bar{\phi} = \sqrt 2 \phi_{\text{sig}}
\]
\[
    \overline{(\phi(t) - \bar{\phi})^2} = \phi^2_{sig}
\]
The electrical signal power is
\[
    \overline{i_{sig}^2} = \left(q \eta \frac{\lambda}{hc}\phi_{sig}\right)^2
\]
\paragraph{Noise}
Can characterize by its variance. If \(v(t)\) is the voltage
\[
    \overline{(\Delta v)^2} = \overline{(v - \bar v)^2}
\]
Typically not interested in phase or offset of the noise (for now).

If the noise is white, and confined to an interval \(\Delta f\), the power of the noise will be proportional to this frequency band. The power of the noise is additive (amplitude is not). The amplitude of the total noise is
\[
    \overline{\Delta v_{tot}} = \sqrt{\sum \overline{(\Delta v_i)2}}
\]
The strongest component tends to dominate.

\paragraph{Parseval's theorem}
We say that
\[
    S(f) = \lvert F(f) \rvert^2
\]
is the power spectrum. Parseval's  states that the integral of the power spectrum is the same as the integral of the squared absolute value of a function (in our case the voltage)

\section{Noise mechanisms}

Photons due not arrive at regular intervals. However, the \emph{probability} that a photon will arrive at a certain point in time, is constant when the photon energy is greater than the mean thermal energy. We can therefore model the arrival of photons with Poisson. The mean number of photons for a time measurement is
\[
    \bar n = \phi_q t_ii
\]
The probability mass function (pmf) becomes, with \(\lambda = \bar n\)
\[
    P(n) = \frac{\bar n ^n}{n!}e^{- \bar n}
\]
Both the variance and mean of this Poisson distributed variable, is \(\bar n\). An important remark is that the photon signal has noise that depends on the photon flux.

Note that while this holds for most practical applications, the variance is wavelength-dependent for the longer wavelengths. It comes from Bose--Einstein statistics. The phenomenon is referred to as \emph{photon bunching}.

\paragraph{Photon noise}

Because the photon rate is not constant, the signal out of a detector won't be, either. An analogy is raindrops falling into a bucket. This problem can be experienced when trying to take a picture in a poorly lit environment,where the numbers of incoming photons will be very small as opposed to a well-lit scene. The photon noise leads to similar noise in the output signal. The expression(s) are similar, but we also need to take the quantum efficiency into account.

When expressed in terms of current, the photon noise leads to what we call \emph{shot noise} in the current
\[
    i_s^2 = \overline{(i - \bar i)^2} = q \frac{\bar i}{t_i}
\]
We can shot that integrating this is equivalent of integrating the power spectrum. We can obtain the relation
\[
    \Delta f = \frac{1}{2 t_i}
\]
between integration time and signal bandwidth. Because the shot noise is white,
\[
    i_s^2 = 2q\bar i \Delta f
\]
The absolute photon noise becomes bigger as the light level increases. However, because the number of photons is larger per integration time interval, the Poisson noise is smaller relative to the signal.

\paragraph{Dark current shot noise}
Must be taken into account. Puts a limit on how weak signals we can detect. Arises due to some unknown/not modeled process in the detector. Similar to the photon noise, we experience this as ``shots''.

\paragraph{Background noise}
Our detector might pick up unwanted signals. For instance, a TV would like to pick up only the IR signal from a remote. However, it will also receive thermal radiation from its surroundings.

\paragraph{Johnson noise}
Noise arising from randomly moving electrons in a circuit. Depends on temperature and resistance.

By constructing an equivalent circuit that gives the same noise in the current, we can compare the noise more easily to the noise in a photocurrent.

\paragraph{Reset noise}
When an integrating capacitor is reset by short circuiting it, its voltage might not be set to zero. In addition, Johnson noise might lead to fluctuations in the capacitor as long as the switch is closed. When the switch is opened, the voltage freezes. This noise is also known as "\(kTC\) noise", as the variance is \(kTC\). The standard deviation is
\[
    \sigma_{\text{reset}} = \sqrt{\frac{kT}{C}}
\]
The noise can be improved/circumvented by using some clever tricks.

\paragraph{Random telegraph noise}
Random switching caused by trapped electrons

\paragraph{\(1/f\) noise}
The observed power spectrum of noise from sensors can often be modeled by
\[
    S_f(f) = \beta_0 \frac{i_0}{f}
\]
where \(i_0\) is the DC bias, and \(\beta_0\) is a scaling factor.

Cause of \(1/f\) noise is largely, unknown. 

\chapter{Noise performance of detectors}

\paragraph{Signal to noise ratio for the incoming photon stream}
\[
    \left(\frac{S}{N}\right)_\text{input} = \frac{\bar n_q}{\sigma_q} = \sqrt{\bar n_q}
\]
The ``true'' signal cannot be estimated better than this, due to inherent randomness of photon arrival.

\paragraph{Signal to noise ratio for the photoelectron signal}
\[
    \left(\frac{S}{N}\right)_\text{output} = \frac{\bar n_e}{\sigma_e} = \sqrt{\eta \bar n_q}
\]
This is the case where the photon statistic noise dominates

\end{document}


